\documentclass{article}

\usepackage{amsmath}
\usepackage{dsfont}
\newcommand{\1}{\mathds{1}}
\usepackage{hyperref}
\usepackage{booktabs}
\usepackage{color}
\newcommand{\red}[1]{{\color{red}#1}}

\title{Disclosure of Controversial Directorships in SEC Filings}
\author{Andrew Marder}
\date{\today}

\begin{document}

\maketitle

\begin{quote}
  Effective February 28, 2010\dots Additional disclosure of any
  directorships held by each director and nominee at any time during
  the past five years at any public company or registered investment
  company.

  \url{https://www.sec.gov/rules/final/2009/33-9089.pdf}
\end{quote}

Did this change in disclosure rules affect how director nominees
report their board memberships in proxy filings?  EDGAR filings
provide board membership data; we know every instance where
individual $i$ is on the board of company $c$ at time $t$. If
individual $i$ is also on the board of company $c'$, we're interested
to see when she does or does not disclose her membership on board
$c'$ in the filing of company $c$. Let $B_i(t)$ be the set of boards that individual $i$ has served
on prior to time $t$. Suppose individual $i$ is nominated for a board
seat for company $c$ at time $t$, she has a choice of whether to
disclose her other board memberships $B_i(t) - \{ c \}$. The data set
we need to construct has a row for each membership that could be
disclosed. Our dependent variable of interest is
\[
y_{ictc'} = \begin{cases}
1 & \text{$i$ disclosed $c'$ membership in filing of company $c$ at time $t$}, \\
0 & \text{otherwise}.
\end{cases}
\]

There are a number of factors we would expect to impact an
individual's choice to disclose board membership:
\begin{enumerate}
\item the reputation of company $c'$ at time $t$,
\item the number of years since individual $i$ was last a member of
  board $c'$,
\item the disclosure policy at time $t$.
\end{enumerate}
Table \ref{tab:vardefs} provides a description of how these factors are
operationalized.

\begin{table}
\centering
\begin{tabular}{lp{3in}}
\toprule
Variable & Description \\
\midrule
$\text{BADREP}_{tc'}$ & company $c'$ has a bad reputation at time $t$ \\
$\text{RECENT}_{itc'}$ & individual $i$'s membership on board $c'$
                      expired less than five years before time $t$ \\
$\text{POST}_t$ & time $t$ falls after policy change - February 28, 2010 \\
$\text{TREAT}_{itc'}$ & $\text{BADREP}_{tc'} * \text{RECENT}_{itc'}$ \\
\bottomrule
\end{tabular}
\caption{\label{tab:vardefs}Variable Definitions}
\end{table}

Using the variable definitions from Table \ref{tab:vardefs}, we use a
difference-in-differences specification to measure the average
treatment effect of the policy change.
\[
y_{ictc'} = \beta_0 + \beta_1 \text{POST}_{t} + \beta_2
\text{TREAT}_{itc'} + \beta_3 \text{POST}_{t}
\text{TREAT}_{itc'} + \epsilon_{ictc'}
\]
The coefficient $\beta_3$ measures the average treatment effect. After the
policy change, how much did reporting of recent directorships on
boards with bad reputations increase?

\section*{Project Road Map}

What needs to be done?
\begin{enumerate}
\item Each biography needs to be extracted from the DEF 14A filings and
  associated with the appropriate nominee.
\item Individuals need to be matched across filings. Individuals can
  be matched using name, birth year, and gender.
\item Board membership disclosures need to be extracted from each
  biography. Looking at individual $i$'s biography in the filing for
  company $c$ we need to see if $i$'s membership in $c'$ is reported.
\item The final data set needs to be constructed and the regression
  model estimated.
\end{enumerate}
Each section of this document describes how I plan to
construct the data needed for each component of this project.

\renewcommand*\contentsname{Outline}
\tableofcontents

\clearpage

\section{Extracting Biographies from Filings}

We have a large number of SEC filings. Let's index those filings using
the variable $i$ which ranges from $1$ to $I$. Each document has a
number of lines, let's index those lines using the variable $t$, which
ranges from $1$ to $T_i$. Notice that the number of lines in a filing
changes from filing to filing. Filing $i$ provides biographies of
$J_i$ directors on the board. Again notice the number of directors
changes across filings. We want to extract the lines in each
filing that discuss a director, and link those lines with the specific
director they discuss. To fit this application into a typical discrete
choice model let
\[
y_{itj} = \begin{cases}
1 & \text{line $t$ in filing $i$ is part of director $j$'s biography,} \\
0 & \text{otherwise.}
\end{cases}
\]
To make each line in filing $i$ about exactly one director we
use $j = 0$ to mark lines that do not belong to director
biographies. If $y_{it0} = 1$ then we know that line $t$ of filing $i$
is not part of a director biography. By setting the data up in this
way we know that 
\[
\sum_{j = 0}^{J_i} y_{itj} = 1
\]
for all $i$ and $t$. In other words, line $t$ in filing $i$ belongs to
exactly one director biography or it belongs to no director
biographies.

Let $x_{itj}$ be a vector of characteristics of line $t$ in filing $i$
with respect to director $j$. We are interested in predicting the
probability that line $t$ is part of director $j$'s biography,
$P(y_{itj} = 1 | x_{itj})$. We have two levers we can use
to improve our predictions: (1) the model we use to estimate the
conditional probabilities and (2) the characteristics we include in the vector of covariates
$x_{itj}$.

\subsection{Model}

To start, I use a conditional logit model to predict the probability
that each line is part of director $j$'s biography:
\[
P(y_{itj} = 1 | x_{itj}) = \frac{\exp(x_{itj} \beta)}{\sum_{j'=0}^{J_i}
\exp(x_{itj'} \beta)}.
\]
I need to set the code up so it is relatively easy to use different
models. That way we can evaluate a variety of models, as it isn't
obvious what model will perform best on this task.

It really seems like we want a nested logit. Thinking about it like a
nested problem will help thinking about setting up the features.

\subsection{Features}

%% https://github.com/jeffdumont/RSGHB/blob/master/Examples/Advanced%20Example%20-%20Nested%20Logit/NL.R#L101

This classification problem has two steps (1) is this line part of a
director biography (2) if it is part of a biography which director is
it about? The second step seems much easier to model. Assuming a line
is part of a biography, the following two variables seem most useful
in assigning that line to a given director:

\begin{itemize}
\item Density of director $j$'s last name in this line.
\item Density of director $j$'s first name in this line.
\end{itemize}

The first step seems a bit harder at the moment.

\begin{itemize}
\item Dummy variable indicating if this row is about no director,
  $\1(j=0)$. It is important to recognize that observations with $j=0$
  are very different from observations with $j>0$. This dummy variable
  may have interesting interactions with other covariates that should
  be considered.
\item More than one director last name contained in this line.
\item No director last name contained in this line.
\item Density of bio words contained in this line.
\end{itemize}

%% TODO: Describe how I chose these words.
\begin{table}[h]
\centering
\begin{tabular}{ll}
Variable & Regular Expression \\
\hline
education & \texttt{(college|university|school|bachelor)} \\
founding & \texttt{(co-)?found(ed|er|ing)?} \\
verb & \texttt{(retired|spent|gained|brings)} \\
pronoun & \texttt{(he|she)} \\
position & \texttt{(chairman|chief|executive|officer|president)} \\
\end{tabular}
\end{table}

\begin{verbatim}
\end{verbatim}

\subsection{Evaluation}

Let $A$ and $B$ be finite sets, the Jaccard index is the ratio of
the number of elements in both sets over the number of elements in
either set
\[
\mathcal{J}(A, B) = \frac{ \| A \cap B \| }{ \| A \cup B \| }.
\]
The Jaccard index ranges from $0$ to $1$ with higher values
representing a higher degree of similarity between the two sets.

Let $A_{ij} = \{ t : y_{itj} = 1 \}$ be the set of lines in document
$i$ about director $j$, and $\hat{A}_{ij}$ be the set of lines in
document $i$ the model predicts to be about director $j$. I evaluate
model predictions using the following metric
\[
\bar{\mathcal{J}}(\hat{A}) = \frac{1}{\sum_{i = 1}^I (J_i + 1)} \sum_{i = 1}^I \sum_{j = 0}^{J_i} \mathcal{J}(A_{ij}, \hat{A}_{ij}).
\]
This metric is the mean Jaccard index across all documents and all
directors. This function returns a single number between $0$ and $1$
describing how similar our predictions match reality.

To avoid over fitting the model, I use k-fold cross validation to
estimate model coefficients with a training set and evaluate the model
fit with a test set.

\subsection{Continuous Improvement}

\[
\begin{array}[c]{ccc}
\text{Corpus} & \red{-\; estimate \rightarrow} & \text{Model Parameters} \\
\red{\uparrow} & & \red{|} \\
\red{update} & & \red{predict} \\
\red{|} & & \red{\downarrow}\\
\text{Problem Children} & \red{\leftarrow identify \;-} & \text{Biographies}
\end{array}
\]

Some thoughts on continuous improvement:

Identifying problem children should be unsupervised.

The update step should allow for modifying the model and manually
coding filings to add to the corpus.

\section{Matching Individuals Across Filings\label{sec:matchit}}

\section{Extracting Directorships from Biographies}

To illustrate this step of the process, I have quoted the first
biography from the first filing in the corpus. It comes from a filing
for the Waters Corporation. I have highlighted Mr. Berthiaume's
disclosed directorship in red. This filing is available
online.\footnote{\url{http://www.sec.gov/Archives/edgar/data/1000697/000095013509002483/b73860dfdef14a.htm}}

\begin{quote}
Douglas A. Berthiaume, 60, has served as Chairman of the Board since
February 1996 and has served as President, Chief Executive Officer and
a Director of the Company since August 1994 (except from January 2002
to March 2003, during which time he did not serve as President). From
1990 to 1994, Mr. Berthiaume served as President of the Waters
Chromatography Division of Millipore Corporation, the predecessor
business of the Company, which was purchased in 1994. Mr. Berthiaume
is the Chairman of the Children's Hospital Trust Board, and a
trustee of the Children's Hospital Medical Center, The
University of Massachusetts Amherst Foundation, and a
director of \red{Genzyme Corporation}.
\end{quote}

There are 7,980 company names in the Equilar database. A naive
comparison of the company names in the database against this biography
would return two matches \texttt{GENZYME CORP} and
\texttt{MILLIPORE CORP /MA}. Notice that interpreting the second match
as a directorship would be a mistake. As an alternative, I propose the
following method to determine whether a directorship is disclosed in a
biography. Suppose we are examining individual $i$'s biography from
company $c$ at time $t$:
\begin{enumerate}
\item Using the table constructed in Section \ref{sec:matchit}, select
  all companies individual $i$ has been a director for with a start
  date prior to time $t$.
\item For each company name use the fuzzywuzzy library to
  calculate the partial ratio of company name to biography text
\[
\text{partial\_ratio}(\text{NAME}_{c'}, \text{BIOGRAPHY}_{ict}).
\]
A value of 100 lets us know the name of company $c'$ occurs in the
biography, a value of 0 lets us know it does not occur, and the closer
the value gets to 100 the more similar a substring of the biography is
to the company name. Record this measure in the database.
\item Plot the distribution of partial ratios across all $ictc'$
  combinations to see if there is an obvious cut point we can use to
  define a disclosed variable
\[
\text{DISCLOSED}_{ictc'} = \text{partial\_ratio}(\text{NAME}_{c'},
\text{BIOGRAPHY}_{ict}) \ge \text{cutoff}.
\]
\end{enumerate}

TODO: This last point is inaccurate. We are going to have RAs mark
whether a filing discloses each directorship expected. We'll use this
data to estimate the optimal cutoff.


\section{Pulling It All Together}

\end{document}
